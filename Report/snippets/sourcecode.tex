%
%  ----------------------------------------------------------------------------
%  	SOURCE CODE SNIPPET with Listing package
%  ----------------------------------------------------------------------------
%	For details see: http://en.wikibooks.org/wiki/LaTeX/Packages/Listings
%   The whole refrence can be found here: http://ftp.univie.ac.at/packages/tex/macros/latex/contrib/listings/listings.pdf
%

%  ----------------------------------------------------------------------------
%  	SOURCE CODE SNIPPET: DEFINITION (put it in preamble e.g.)
%  ----------------------------------------------------------------------------

\usepackage{listings}


%  ----------------------------------------------------------------------------
%  	SOURCE CODE SNIPPET: Possible configuration
%  ----------------------------------------------------------------------------

% Add line numbers
\lstset{numbers=left, numberstyle=\tiny, stepnumber=2, numbersep=5pt}

% Add round border 
\lstset{frameround=fttt}

% Change Background
\lstset{backgroundcolor=\color{yellow}}

% Set language: language=[<dialect>]<language>
\lstset{language=java}

% Frame argument examples:
%   [frame=single] - border on all 4 sides
%   [frame=tb] - single border on top and bottom
%   [frame=trBL] - single border on top and right, double border on bottom and left


% a more complete example
\lstset{ %
	basicstyle=\footnotesize,       % the size of the fonts that are used for the code
	numbers=left,                   % where to put the line-numbers
	numberstyle=\footnotesize,      % the size of the fonts that are used for the line-numbers
	stepnumber=2,                   % the step between two line-numbers. If it's 1, each line 
	                                % will be numbered
	numbersep=5pt,                  % how far the line-numbers are from the code
	backgroundcolor=\color{white},  % choose the background color. You must add \usepackage{color}
	showspaces=false,               % show spaces adding particular underscores
	showstringspaces=false,         % underline spaces within strings
	showtabs=false,                 % show tabs within strings adding particular underscores
	frame=single,                   % adds a frame around the code
	tabsize=2,                      % sets default tabsize to 2 spaces
	captionpos=b,                   % sets the caption-position to bottom
	breaklines=true,                % sets automatic line breaking
	breakatwhitespace=false,        % sets if automatic breaks should only happen at whitespace
	title=\lstname,                 % show the filename of files included with \lstinputlisting;
	                                % also try caption instead of title
	escapeinside={\%*}{*)},         % if you want to add a comment within your code
	morekeywords={*,...}            % if you want to add more keywords to the set
}

%  ----------------------------------------------------------------------------
%  	SOURCE CODE SNIPPET: EXAMPLES
%  ----------------------------------------------------------------------------

\begin{lstlisting}[caption={This is a caption to the code},label=CodeA, frame=tb]
public class ServerStarter {
	public static void main(String[] arg0) throws Exception {
		
		RestServer s = new RestServer("127.0.0.1", 9000);
		/* starts the server */
		s.startServer();
		
		//...
		
		/* stops the server */
		s.stopServer();
	}
}
\end{lstlisting}