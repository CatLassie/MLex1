%  ----------------------------------------------------------------------------
%
%       Copyright (for the template) 2011 by [pfavre]
%
%  ----------------------------------------------------------------------------
%  CONTENT
%  ----------------------------------------------------------------------------
%	

\section{Kurzes Tutorial}
Das ist eine simples \LaTeX-Beispiel. Mit dem Hyperref package können Referenzen in einem PDF geklickt werden. Man kann natürlich auch Fußnoten setzten\footnote{Welches hier unten gelesen werden kann}.

Mit einer Leerzeile im Editor entsteht ein neuer Absatz. Ich kann natürlich auch eine Aufzählung machen:
\begin{itemize}
  \item Das erste Element
  \item Das zweite Element
  \item und so weiter
  \begin{itemize}
	  \item Unterelement
	  \item sind natürlich
	  \item auch möglich
	\end{itemize}
\end{itemize}


Jetzt zeige ich eine Referenz zu Abschnitt \ref{sec:Fonts Bearbeiten} . Damit das funktioniert muss man manuell Labels unter sections setzten. Mit dem Hyperref package können Referenzen in einem PDF geklickt werden.

Eine andere Art Dinge aufzulisten ist die Definitionsliste:
\begin{description}
	\item[Definition1] Erklärung
	\item[Definition2] Erklärung
\end{description}
\subsection{Tabellen}
Dies ist eine normale Tabelle
\begin{tabular}[h]{|l|c|} %[Platzierung]{Spalten und deren align und borders}
  \hline
  Lehrstuhl & Professoren\\
  \hline\hline
  LS2 & Wegener\\
  LS8 & Morik\\
  LS10 & Doberkat\\
  \hline
\end{tabular}

\subsection{Weitere Funktionen}
Man kann das Dokument mit sehr vielen Funktionen nutzen wie zB. Appendix, Stichwortverzeichnis, Glossar, Literaturverzeichnis und und und.

\subsubsection{Fonts Bearbeiten}
\label{sec:Fonts Bearbeiten}
Wenn ich möchte kann ich den Text auch \textbf{Fett} machen oder \textit{Kursiv} machen. Es gibt außerdem die Möglichkeit zwischen \textrm{serifen} und \textsf{serifenlosen} Schriften zu wählen sowie eine \texttt{Typwriter Font} zu benutzen. Da serifen Schrift die Standard Schriftart ist sieht man hier keinen Unterschied.
\subsubsection{Bilder}
Man kann natürlich auch Bilder einfügen wie man das in Abbildung \ref{fig:TUWien Logo} sehr gut sehen kann. Da \LaTeX \; die Position des Bildes selber bestimmt, kann es sein, dass es nicht direkt unter der Überschrift ist.

\begin{figure}
	\centering
	\includegraphics[scale=0.3]{img/logo_tuwien_with_informatics.png}
	\caption{Eine einfache Bildunterschrift}
	\label{fig:TUWien Logo}
\end{figure}

\section{Zu diesem Template}
Ich habe in der Datei pages/config.tex einige Variablen definiert welche dann im Titelblatt und Fuß- unf Kopfzeile ersetzt werden. Außerdem kann man mit einigen boolean Werten gewisse Teile des Dokuments aus und einschalten.


