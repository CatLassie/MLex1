\section{Datasets}
We used the following datasets:
\begin{itemize}
\item Congress (Kaggle)
\item Amazon (Kaggle)
\item Lung Cancer
\end{itemize}

\section{Experiments on Congress}
No difference between gini and Information gain for Decision Tree when using all features

physician-fee-freeze seems to be important as removing it from the features decreased metrics quite drastically

Surprisingly, adding or removing education-spending from the features did not make a significant difference.

Removing feature 'superfund-right-to-sue' slightly improved results for DT but no impact for NB.

\subsection{SVM}

rbf kernel
[[ 78   3]
 [  9 128]]
128 3 78 9
0.94495412844 0.94495412844 0.977099236641 2.0

linear kernel
[[ 78   3]
 [  6 131]]
131 3 78 6
0.95871559633 0.95871559633 0.977611940299 2.0

poly
[[ 77   4]
 [ 12 125]]
125 4 77 12
0.926605504587 0.926605504587 0.968992248062 2.0

sigmoid
[[ 76   5]
 [  9 128]]
128 5 76 9
0.935779816514 0.935779816514 0.962406015038 2.0

Further experiments for linear kernel
Lowering C -> negative effect
Changing decision function: no effect

\subsection{Random Forest}
Lower number of trees better results?

\section{Experiments on Amazon}
\subsection{SVM}
SVC linear
[[ 9  0  0 ...,  0  0  0]
 [ 0 11  1 ...,  0  0  0]
 [ 0  0 17 ...,  0  0  0]
 ..., 
 [ 0  0  0 ...,  9  0  0]
 [ 0  0  0 ...,  0  7  0]
 [ 0  0  0 ...,  0  0 16]]
11 0 9 0
0.50875 0.50875 0.50875 2.0

SVC sigmoid

[[ 0  0  7 ...,  0  0  0]
 [ 0  0  9 ...,  0  0  0]
 [ 0  0  5 ...,  0  0  0]
 ..., 
 [ 0  0 10 ...,  0  0  0]
 [ 0  0  5 ...,  0  0  0]
 [ 0  0  9 ...,  0  0  0]]
0 0 0 0
0.01 0.01 0.01 2.0

SVC rbf
[[ 7  0  0 ...,  0  0  0]
 [ 0 10  0 ...,  0  0  0]
 [ 0  0 15 ...,  0  0  0]
 ..., 
 [ 0  0  0 ...,  6  0  0]
 [ 0  0  0 ...,  0  1  0]
 [ 0  0  0 ...,  0  0  8]]
10 0 7 0
0.28125 0.28125 0.28125 2.0

\section{Cancer}
\url{https://archive.ics.uci.edu/ml/datasets/Lung+Cancer}

\subsection{Preprocessing}
Two imputation strategies: imputing value as col-mean and imputing it as mean of the class. Yielded no significant difference for KNeighbors.
However, this might be due to the small size of the dataset. 
Additionally, when removing the five samples with missing values the performance of the algorithms increased. 

\section{KDD Cup 99}
\url{https://archive.ics.uci.edu/ml/datasets/KDD+Cup+1999+Data}