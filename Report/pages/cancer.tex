\section{Cancer}
This dataset is interesting for one aspect: it has no descriptive information i.e. neither the classes nor the columns have meaningful attributes. 
Literature reports best results for kNN algorithms. However, we found better results for SVMs which would contradict these results.
However, the SVM could not be validated against unseen data. 
Moreover, the are a discrepancies with missing values, as will be described in the next section. 
Throughout our experiments it became clear that type $2$ is hard to distinguish from $1$ and $2$ as the its corresponding rows in the confusion matrices showed more values than just in the diagonal. Additionally, it seems that there is a slight tendency that $2$ is misclassified as $1$ for our tested algorithms and parameter combinations. 

\subsection{Preprocessing}
Two imputation strategies: imputing value as col-mean and imputing it as mean of the class. Yielded no significant difference for KNN.
However, this might be due to the small size of the dataset. 
Additionally, when removing the five samples with missing values the performance (accuracy) of most algorithms increased. 

For the experiments we imputed missing values using the class average. 

\subsection{GNB}
The execution time for GNB was $1.73 \pm 0.10$ ms with an average accuracy of $0.47 \pm 0.17$.

\subsection{Decision Tree \& Random Forest}
In our experiments we tried different combinations for the criterion (Information Gain vs Gini Impurity) and maximum features as well as the maximal number of estimators for random forests.
The experiments yielded the best performance for decision trees when using log while random forests performed best when using sqrt. 

\subsection{SVM}
Our initial tests included the following kernels: 'linear','sigmoid', 'rbf', 'poly'. 
In our experiments we were unable to tune the 'gamma' parameter in order to obtain better results than those obtained when using the default parameter.
The best results for the SVM using a polynomial kernel with $C=11$ an accuracy of $0.53 \pm 0.17$ using $0.58 \pm 0.01$ ms as runtime. 

\subsection{kNN}
We experimented with different parameters and varied the neighbours from $1$ to $20$. 
Additionally, the $2x2$ combinations for the weights and distance functions. 
In our experiments lower values for neighbours yielded better results. 
The best results were obtained using $5$ neighbours with uniform weights and euclidean distance which yielded an accuracy of $0.59 \pm 0.30$ using $1.22 \pm 0.10$ ms runtime. 
An evaluation of some tested parameters is outlined in table~\ref{tbl:cancer.knn} and while confusion matrix for our final result is outlined in table~\ref{tbl:cancer.knn.cm}.

\begin{table}[htb]
\centering
\begin{tabular}{cccrr}
\toprule
\textbf{p} & \textbf{weights} & \textbf{n\_neighbors} & \textbf{time (ms)} & \textbf{score}\\
\midrule
1 & uniform & 4 & $3.94 \pm 7.60$ & $0.41 \pm 0.15$\\
1 & distance & 4 & $1.50 \pm 0.21$ & $0.44 \pm 0.27$\\
1 & uniform & 5 & $1.38 \pm 0.04$ & $0.44 \pm 0.26$\\
1 & distance & 5 & $1.37 \pm 0.06$ & $0.44 \pm 0.25$\\
1 & uniform & 6 & $1.41 \pm 0.05$ & $0.44 \pm 0.25$\\
1 & distance & 6 & $1.46 \pm 0.13$ & $0.50 \pm 0.30$\\
1 & uniform & 7 & $1.41 \pm 0.06$ & $0.47 \pm 0.31$\\
1 & distance & 7 & $1.39 \pm 0.05$ & $0.50 \pm 0.35$\\
2 & uniform & 4 & $1.43 \pm 0.05$ & $0.44 \pm 0.19$\\
2 & distance & 4 & $1.40 \pm 0.09$ & $0.50 \pm 0.22$\\
2 & uniform & 5 & $1.38 \pm 0.03$ & $0.59 \pm 0.30$\\
2 & distance & 5 & $1.37 \pm 0.02$ & $0.50 \pm 0.27$\\
2 & uniform & 6 & $1.41 \pm 0.08$ & $0.44 \pm 0.27$\\
2 & distance & 6 & $1.45 \pm 0.06$ & $0.53 \pm 0.30$\\
2 & uniform & 7 & $1.43 \pm 0.08$ & $0.59 \pm 0.33$\\
2 & distance & 7 & $1.41 \pm 0.03$ & $0.56 \pm 0.32$\\
\bottomrule
\end{tabular}

\caption{Cancer Results using kNN}
\label{tbl:cancer.knn}
\end{table}

\begin{table}[hbt]
\centering
\begin{tabular}{l|ccc}
\toprule
&\textbf{1} & \textbf{2} & \textbf{3}\\
\midrule
\textbf{1} & 7 & 2 & 0\\
\textbf{2} & 7 & 4 & 2\\
\textbf{3} & 0 & 7 & 3\\
\bottomrule
\end{tabular}

\caption{Confusion Matrix for Cancer - kNN}
\label{tbl:cancer.knn.cm}
\end{table}
