\section{Experiments on Congress}
No difference between gini and Information gain for Decision Tree when using all features

physician-fee-freeze seems to be important as removing it from the features decreased metrics quite drastically

Surprisingly, adding or removing education-spending from the features did not make a significant difference.

Removing feature 'superfund-right-to-sue' slightly improved results for DT but no impact for NB.

\subsection{SVM}

rbf kernel
[[ 78   3]
 [  9 128]]
128 3 78 9
0.94495412844 0.94495412844 0.977099236641 2.0

linear kernel
[[ 78   3]
 [  6 131]]
131 3 78 6
0.95871559633 0.95871559633 0.977611940299 2.0

poly
[[ 77   4]
 [ 12 125]]
125 4 77 12
0.926605504587 0.926605504587 0.968992248062 2.0

sigmoid
[[ 76   5]
 [  9 128]]
128 5 76 9
0.935779816514 0.935779816514 0.962406015038 2.0

Further experiments for linear kernel
Lowering C -> negative effect
Changing decision function: no effect

\subsection{KNN}
\begin{table}[tb]
\centering
\begin{tabular}{cccrr}
\toprule
\textbf{p} & \textbf{weights} & \textbf{n\_neighbors} & \textbf{time (ms)} & \textbf{score}\\
\midrule
1 & uniform & 4 & $1.61 \pm 0.08$ & $0.93 \pm 0.05$\\
1 & distance & 4 & $1.61 \pm 0.13$ & $0.93 \pm 0.04$\\
1 & uniform & 5 & $1.58 \pm 0.05$ & $0.93 \pm 0.05$\\
1 & distance & 5 & $1.55 \pm 0.04$ & $0.93 \pm 0.05$\\
1 & uniform & 6 & $1.55 \pm 0.03$ & $0.92 \pm 0.05$\\
1 & distance & 6 & $1.55 \pm 0.03$ & $0.93 \pm 0.05$\\
1 & uniform & 7 & $1.56 \pm 0.06$ & $0.94 \pm 0.04$\\
1 & distance & 7 & $1.59 \pm 0.11$ & $0.94 \pm 0.04$\\
2 & uniform & 4 & $1.58 \pm 0.04$ & $0.92 \pm 0.05$\\
2 & distance & 4 & $1.58 \pm 0.06$ & $0.92 \pm 0.05$\\
2 & uniform & 5 & $1.57 \pm 0.10$ & $0.94 \pm 0.04$\\
2 & distance & 5 & $1.59 \pm 0.11$ & $0.94 \pm 0.04$\\
2 & uniform & 6 & $1.57 \pm 0.05$ & $0.92 \pm 0.05$\\
2 & distance & 6 & $1.57 \pm 0.04$ & $0.94 \pm 0.04$\\
2 & uniform & 7 & $1.58 \pm 0.04$ & $0.92 \pm 0.04$\\
2 & distance & 7 & $1.54 \pm 0.02$ & $0.92 \pm 0.04$\\
\bottomrule
\end{tabular}

\caption{Congress - KNN}
\end{table}

\subsection{Random Forest}
Lower number of trees better results?
\begin{table}[tb]
\centering
\begin{tabular}{cccrr}
\toprule
\textbf{criterion} & \textbf{n\_estimators} & \textbf{max\_features} & \textbf{time (ms)} & \textbf{score}\\
\midrule
entropy & 10 & auto & $14.17 \pm 0.30$ & $0.97 \pm 0.03$\\
entropy & 11 & auto & $15.10 \pm 0.29$ & $0.96 \pm 0.03$\\
entropy & 12 & auto & $16.34 \pm 0.60$ & $0.95 \pm 0.03$\\
entropy & 13 & auto & $17.24 \pm 0.34$ & $0.95 \pm 0.04$\\
entropy & 14 & auto & $18.33 \pm 0.23$ & $0.96 \pm 0.04$\\
entropy & 15 & auto & $19.89 \pm 0.39$ & $0.96 \pm 0.03$\\
entropy & 16 & auto & $21.25 \pm 0.63$ & $0.94 \pm 0.04$\\
entropy & 17 & auto & $22.50 \pm 0.91$ & $0.96 \pm 0.04$\\
entropy & 18 & auto & $23.28 \pm 0.32$ & $0.95 \pm 0.03$\\
entropy & 19 & auto & $24.38 \pm 0.47$ & $0.96 \pm 0.03$\\
entropy & 10 & sqrt & $14.08 \pm 0.59$ & $0.94 \pm 0.03$\\
entropy & 11 & sqrt & $15.82 \pm 0.38$ & $0.95 \pm 0.03$\\
entropy & 12 & sqrt & $17.07 \pm 1.25$ & $0.96 \pm 0.04$\\
entropy & 13 & sqrt & $18.19 \pm 0.26$ & $0.96 \pm 0.03$\\
entropy & 14 & sqrt & $18.95 \pm 0.57$ & $0.95 \pm 0.04$\\
entropy & 15 & sqrt & $19.75 \pm 0.39$ & $0.96 \pm 0.03$\\
entropy & 16 & sqrt & $21.31 \pm 0.45$ & $0.96 \pm 0.04$\\
entropy & 17 & sqrt & $22.50 \pm 0.92$ & $0.96 \pm 0.03$\\
entropy & 18 & sqrt & $22.99 \pm 0.25$ & $0.96 \pm 0.04$\\
entropy & 19 & sqrt & $24.10 \pm 0.40$ & $0.95 \pm 0.05$\\
entropy & 10 & log2 & $14.28 \pm 0.97$ & $0.95 \pm 0.06$\\
entropy & 11 & log2 & $15.55 \pm 0.39$ & $0.95 \pm 0.04$\\
entropy & 12 & log2 & $16.43 \pm 0.83$ & $0.95 \pm 0.04$\\
entropy & 13 & log2 & $17.54 \pm 0.31$ & $0.96 \pm 0.03$\\
entropy & 14 & log2 & $18.36 \pm 0.39$ & $0.94 \pm 0.04$\\
entropy & 15 & log2 & $19.69 \pm 0.35$ & $0.96 \pm 0.03$\\
entropy & 16 & log2 & $22.13 \pm 1.17$ & $0.97 \pm 0.04$\\
entropy & 17 & log2 & $23.32 \pm 1.64$ & $0.96 \pm 0.04$\\
entropy & 18 & log2 & $23.23 \pm 0.48$ & $0.96 \pm 0.03$\\
entropy & 19 & log2 & $24.16 \pm 0.24$ & $0.96 \pm 0.03$\\
gini & 10 & auto & $13.65 \pm 0.26$ & $0.94 \pm 0.03$\\
gini & 11 & auto & $15.04 \pm 0.79$ & $0.97 \pm 0.04$\\
gini & 12 & auto & $16.16 \pm 0.38$ & $0.96 \pm 0.04$\\
gini & 13 & auto & $16.96 \pm 0.45$ & $0.96 \pm 0.03$\\
gini & 14 & auto & $18.06 \pm 0.19$ & $0.95 \pm 0.04$\\
gini & 15 & auto & $20.42 \pm 1.18$ & $0.95 \pm 0.05$\\
gini & 16 & auto & $20.85 \pm 0.83$ & $0.95 \pm 0.03$\\
gini & 17 & auto & $22.18 \pm 0.67$ & $0.96 \pm 0.04$\\
gini & 18 & auto & $23.03 \pm 0.59$ & $0.96 \pm 0.03$\\
gini & 19 & auto & $24.09 \pm 0.96$ & $0.95 \pm 0.04$\\
gini & 10 & sqrt & $14.12 \pm 0.38$ & $0.96 \pm 0.03$\\
gini & 11 & sqrt & $14.69 \pm 0.38$ & $0.96 \pm 0.04$\\
gini & 12 & sqrt & $16.47 \pm 0.93$ & $0.95 \pm 0.03$\\
gini & 13 & sqrt & $17.69 \pm 0.55$ & $0.95 \pm 0.03$\\
gini & 14 & sqrt & $18.50 \pm 0.67$ & $0.95 \pm 0.03$\\
gini & 15 & sqrt & $19.99 \pm 0.77$ & $0.96 \pm 0.04$\\
gini & 16 & sqrt & $20.62 \pm 0.49$ & $0.96 \pm 0.03$\\
gini & 17 & sqrt & $21.96 \pm 1.06$ & $0.96 \pm 0.03$\\
gini & 18 & sqrt & $22.51 \pm 0.12$ & $0.97 \pm 0.04$\\
gini & 19 & sqrt & $23.81 \pm 0.75$ & $0.96 \pm 0.03$\\
gini & 10 & log2 & $13.79 \pm 0.91$ & $0.96 \pm 0.03$\\
gini & 11 & log2 & $15.02 \pm 0.43$ & $0.95 \pm 0.04$\\
gini & 12 & log2 & $16.61 \pm 0.57$ & $0.95 \pm 0.04$\\
gini & 13 & log2 & $17.15 \pm 0.50$ & $0.96 \pm 0.03$\\
gini & 14 & log2 & $18.32 \pm 0.47$ & $0.95 \pm 0.04$\\
gini & 15 & log2 & $19.20 \pm 0.23$ & $0.96 \pm 0.04$\\
gini & 16 & log2 & $21.11 \pm 0.65$ & $0.95 \pm 0.04$\\
gini & 17 & log2 & $22.11 \pm 1.20$ & $0.94 \pm 0.03$\\
gini & 18 & log2 & $22.82 \pm 0.54$ & $0.96 \pm 0.03$\\
gini & 19 & log2 & $23.94 \pm 0.40$ & $0.95 \pm 0.04$\\
\bottomrule
\end{tabular}

\caption{Congress - Random Forest}
\end{table}
